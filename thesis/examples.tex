\chapter{Restrict by example}\label{chapter:examples}
In chapter \ref{chap:introduction} we have already seen a simple example of an optimization permitted by restrict.
To get a better intuition for the applications of restrict, we will look at some more example programs in this chapter.
In section \ref{section:example-redundant-load} and section \ref{section:example-redundant-store} we show how programs which
perform redundant memory operations (loads and stores) can be optimized using restrict.
In section \ref{section-example-memcpy} we show how a potential implementation of C's standard library function
\textcode{memcpy} specialized for integers can benefit from using restrict.
Finally, section \ref{section:iso-definition} discusses the official definition of restrict in
the ISO/IEC C11 standard.

Sections \ref{section:example-redundant-load} and \ref{section:example-redundant-store} use
x86 assembly code whereas section \ref{section-example-memcpy} uses RISC-V (specifically, RV32IV) assembly code.
This distinction is made to simplify explaining the effect of restrict on a program,
but not important for permitting optimizations (\ie all compiled programs could also have been shown
in the same assembly language).

\section{Redundant load}\label{section:example-redundant-load}
Consider the function \textcode{foo}, taken from Fasselt's report on C keywords \cite{fasselt2014keywords},
defined on the left in figure \ref{lst:example-redundant-load}.
This function takes three integer pointers $a, b$ and $c$ as arguments, of which only pointer $c$ is restrict qualified.
The body of \textcode{foo} consecutively adds the value of $c$'s pointee (the object referred to by $c$) to the pointees of $a$ and $b$.    

The first observation we make is that the pointer expression $c$ is based on the restrict qualified object $c$ and used to load (\ie an \textit{access}) from its pointee, which we call $X$.
Because $X$ is accessed through $c$ we know that if $X$ gets modified then all accesses must happen through pointer expressions also based on $c$.
Secondly, the pointer expressions $a$ and $b$ both store to their pointees (\ie they \textit{modify} the objects they point to), and are \textbf{not} based on $c$.
A compiler may therefore conclude that both $a$ and $b$ do not alias with $c$.

With this information, the compiler knows that the store at line \ref{lst:example-redundant-load-line-store} does not change the value of $X$.
When the value of $X$ is required again at line \ref{lst:example-redundant-load-line-load},
it can simply reuse the previously fetched value instead of performing a new memory load.
The x86 assembly produced by GCC and Clang when compiling this program without optimizations is shown on the right.
Line \ref{lst:example-redundant-load-assembly} of the assembly corresponds to the second load of $X$.
When compiled with optimization flag -O3 this line is omitted by both compilers.
This optimization does not happen without restrict, which shows that it actually permits the desired optimization.

\begin{figure*}[!h]
\centering
\begin{minipage}[t]{0.53\linewidth}
\centering
\begin{code}
\begin{minted}[escapeinside=||,mathescape=true,linenos]{c}
void foo (int* a, int* b, int* restrict c)
{
    *a += *c; // First load of $c$ $\label{lst:example-redundant-load-line-store}$
    *b += *c; // Redundant reload from $c$ $\label{lst:example-redundant-load-line-load}$
}
\end{minted}
\end{code}
\end{minipage}%
\begin{minipage}[t]{0.47\linewidth}
\centering
\begin{code}
\centering
\begin{minted}[escapeinside=||,mathescape=true,linenos]{nasm}
foo:
mov     eax, DWORD PTR [rdx]
add     DWORD PTR [rdi], eax
|\colorbox{gray}{mov     eax, DWORD PTR [rdx]}| # Redundant $\label{lst:example-redundant-load-assembly}$
add     DWORD PTR [rsi], eax
ret
\end{minted}
\end{code}
\end{minipage}
\captionof{listing}{Redundant memory load}
\label{lst:example-redundant-load}
\end{figure*}

\section{Redundant store}\label{section:example-redundant-store}
Another optimization permitted by restrict which a compiler may perform is the removal of redundant memory stores.
Consider the function \textcode{foo}, inspired by Jung \etall's \textcode{example3\_down} function
in Stacked Borrows \cite{jung2019stacked}, on the left in figure \ref{lst:example-redundant-store}.
This function takes a restrict qualified pointer $p$ as argument.
At line \ref{lst:example-redundant-store-redundant-store} the value 10 is stored into
$p$'s pointee, which we call $X$.
Because $X$ gets modified, all pointer expressions which also access $X$ must also 
be based on $p$.
The call to the externally defined function \textcode{bar} does not get any arguments passed,
which means that it is not allowed to access $X$!
This means that when executing line \ref{lst:example-redundant-store-overwritten} the first store
becomes redundant: it is overwritten without its value having been loaded in between.

The x86 assembly produced by GCC and Clang when compiling this program without optimizations is shown on 
the right.
Line \ref{lst:example-redundant-store-assembly} of the assembly corresponds to the first store to $X$.
When compiled with optimization flag -O3 this line is omitted by Clang, which does not happen when restrict is 
not present.
GCC does not perform this optimization, which is fine: an implementor \textit{may} use restrict to perform
more optimizations, but performing an optimization is by no means required.
In fact, the ISO/IEC definition explicitly mentions that an implementor may ignore all uses of restrict.

% Inspired by Stacked Borrows example3\_down 
% Clang: yes, GCC: no

\begin{figure*}[!h]
\centering
\begin{minipage}[t]{0.5\linewidth}
\centering
\begin{code}
\begin{minted}[escapeinside=||,mathescape=true,linenos]{c}
extern void bar();

void foo(int* restrict p)
{
    *p = 10; // Redundant store $\label{lst:example-redundant-store-redundant-store}$
    bar();
    *p = 11; // Overwritten $\label{lst:example-redundant-store-overwritten}$
}
\end{minted}
\end{code}
\end{minipage}%
\begin{minipage}[t]{0.5\linewidth}
\centering
\begin{code}
\centering
\begin{minted}[escapeinside=||,mathescape=true,linenos]{nasm}
foo:
push    rbx
xor     eax, eax
mov     rbx, rdi
|\colorbox{gray}{mov     DWORD PTR [rdi], 10}| # Redundant $\label{lst:example-redundant-store-assembly}$
call    bar
mov     DWORD PTR [rbx], 11
pop     rbx
ret
\end{minted}
\end{code}
\end{minipage}
\captionof{listing}{Redundant memory store}
\label{lst:example-redundant-store}
\end{figure*}

\section{Memcpy}\label{section-example-memcpy}
% https://godbolt.org/z/WjszYY5vK 
% X86 GCC flags: -O3 -march=skylake-avx512 -ffreestanding

% https://godbolt.org/z/KfP74xfhW
% RISC-V rv32gc clang 14.0.0
% flags: -O3 -ffreestanding -fno-strict-aliasing -march=rv32gcv -target riscv32 -mllvm --riscv-v-vector-bits-min=256


In Listing \ref{Lst:copy} a possible implementation for a function which
copies $n$ integers from \texttt{src} to \texttt{dst} is given.
For architectures which have vector operations at their disposal, a modern compiler typically wants to utilize these operations
to efficiently load the \texttt{src} array and store the retrieved data in the \texttt{dst} array.

LLVM (the compiler backend of Clang among others) is one such compiler, which has an ``auto vectorizer" looking to vectorize parts of the code where possible.
When we compile the copy function using Clang 14.0.0, the generated RV32IV assembly code indeed shows that vector instructions are utilized.

However, if \texttt{src} and \texttt{dst} point to overlapping regions in memory, such optimizations may not always be performed.
An example invocation (with $n=8$) which could potentially produce an ``incorrect'' output
is depicted in figure \ref{Fig:memoryprogressionscopy}: here the destination array has overlap with the source array
because the \texttt{dst} pointer has the value \texttt{src} + 2.  

Figure \ref{Fig:memorycopy1} depicts an element-wise copy: after two copies
the incremented \texttt{src} pointer lies within previously changed data, and thus the next copy
will now refer to this ``new" data. After eight copies, the destination array has become
\texttt{[1,2,1,2,1,2,1,2]}.

Figure \ref{Fig:memorycopy2} depicts a vectorized copy, with the vector size $= n = 8$.
Here, the generated code will first load all $8$ integer values from \texttt{src}, and
then store them with a single instruction in \texttt{dst}. The destination array has become
\texttt{[1,2,3,4,5,6,7,8]}. 

\newpage

\begin{code}
\begin{minted}[escapeinside=||,mathescape=true,linenos]{c}
    void copy(int* /*restrict*/ src, int* /*restrict*/ dst, int n) {
        for (int i = 0; i < n; ++i) {
            dst[i] = src[i];
        }
    }
\end{minted}
\captionof{listing}{A possible implementation of \texttt{copy}}
\label{Lst:copy}
\end{code}

\vspace*{2em}

\tikzset{pointer/.style={node distance=3em, text height=0.3ex,text depth=.25ex},
         cellcopy/.style={gray, dashed, ->, shorten >=1pt, node distance=2mm},
         cell/.style=   {minimum size=4ex,node distance=-0.14mm}}

\begin{figure}[h]
\centering
\begin{subfigure}[b]{\textwidth}
\begin{tikzpicture}[
    start chain=src going right,
    start chain=dst1 going right,
    start chain=dst2 going right
]
    \tikzstyle{every node}=[font=\small]

    % src array
    \foreach \x in {1,2,...,8,,,\ldots} {
        \node [cell,draw,on chain=src] {\x};
    }

    \node [on chain = src] {\texttt{(I) Initial memory state}};

    \node[pointer, above of=src-3] (dstPointer) {dst};
    \node[pointer, above of=src-1] (srcPointer1) {src};
    \draw[->, shorten >=1pt] (dstPointer) -- (src-3);
    \draw[->, shorten >=1pt, rounded corners] (srcPointer1) -- (src-1);

    % dst array 1; start with the first node to apply
    % the y-shift and then continue right
    \node [cell,draw, fill=white, on chain=dst1, yshift=-15ex] {1};

    \foreach \x/\col in {2/white,1/gray!30,2/gray!30,5/white,6/white,7/white,8/white,/white,/white,\ldots/white} {
        \node [cell,draw, fill=\col, on chain=dst1] {\x};
    }

    \node [on chain = dst1,text width=6cm] {\texttt{(II) Memory state after two individual copies}};

    \node[pointer, above of=dst1-5] (dstPointer) {dst};
    \node[pointer, above of=dst1-3, color=red] (srcPointer2) {src + 2};
    \draw[->, shorten >=1pt] (dstPointer) -- (dst1-5);
    \draw[->, red, shorten >=1pt, fill] (srcPointer2) -- (dst1-3);

    \draw [cellcopy] (src-1) -- (dst1-3);
    \draw [cellcopy] (src-2) -- (dst1-4);

    % dst array 2; start with the first node to apply
    % the y-shift and then continue right
    \node [cell,draw, fill=white, on chain=dst2, yshift=-30ex] {1};

    \foreach \x/\col in {2/white,1/white,2/white,1/gray!30,2/gray!30,1/gray!30,2/gray!30,1/gray!30,2/gray!30,\ldots/white} {
        \node [cell,draw, fill=\col, on chain=dst2] {\x};
    }

    \node [on chain = dst2,text width=5cm] {\texttt{(III) Memory state after copy was executed}};

    \node[pointer, above of=dst2-10] (dstPointer) {dst};
    \node[pointer, above of=dst2-8, color=red] (srcPointer3) {src + 7};
    \draw[->, shorten >=1pt] (dstPointer) -- (dst2-10);
    \draw[->, red, shorten >=1pt, fill] (srcPointer3) -- (dst2-8);

    \draw [cellcopy] (dst1-3) -- (dst2-5);
    \draw [cellcopy] (dst1-4) -- (dst2-6);
    \draw [cellcopy] (dst1-5) -- (dst2-7);
    \draw [cellcopy] (dst1-6) -- (dst2-8);
    \draw [cellcopy] (dst1-7) -- (dst2-9);
    \draw [cellcopy] (dst1-8) -- (dst2-10);

\end{tikzpicture}
\caption{Memory progression of element-wise copy $n=8$}
\label{Fig:memorycopy1}
\end{subfigure}
\par\bigskip\bigskip\bigskip
\begin{subfigure}[b]{\textwidth}
    \begin{tikzpicture}[
        start chain=src going right,
        start chain=dst going right
    ]
        \tikzstyle{every node}=[font=\small]
    
        % src array
        \foreach \x in {1,2,...,8,,,\ldots} {
            \node [cell,draw,on chain=src] {\x};
        }
    
        \node [on chain = src] {\texttt{(I) Initial memory state}};
    
        \node[pointer, above of=src-3] (dstPointer) {dst};
        \node[pointer, above of=src-1] (srcPointer1) {src};
        \draw[->, shorten >=1pt] (dstPointer) -- (src-3);
        \draw[->, shorten >=1pt, rounded corners] (srcPointer1) -- (src-1);

        % dst array
        \node [cell,draw, fill=white, on chain=dst, yshift=-15ex] {1};
        \node [cell,draw, fill=white, on chain=dst] {2};

        \foreach \x in {1,2,...,8} {
            \node [cell,draw, fill=gray!30, on chain=dst] {\x};
        }

        \node [cell,draw, fill=white, on chain=dst] {\ldots};
        \node [on chain = dst, text width=5cm] {\texttt{(II)  Memory state after vectorized copy was executed}};

        \node[pointer, above of=dst-10] (dstPointer) {dst};
        \node[pointer, above of=dst-8, color=red] (srcPointer) {src + 7};
        \draw[->, shorten >=1pt] (dstPointer) -- (dst-10);
        \draw[->, red, shorten >=1pt, fill] (srcPointer) -- (dst-8);

        \draw [cellcopy] (src-1) -- (dst-3);
        \draw [cellcopy] (src-2) -- (dst-4);
        \draw [cellcopy] (src-3) -- (dst-5);
        \draw [cellcopy] (src-4) -- (dst-6);
        \draw [cellcopy] (src-5) -- (dst-7);
        \draw [cellcopy] (src-6) -- (dst-8);
        \draw [cellcopy] (src-7) -- (dst-9);
        \draw [cellcopy] (src-8) -- (dst-10);

\end{tikzpicture}
\caption{Memory progression of vector copy $n=8$} \label{fig:copy2}
\label{Fig:memorycopy2}
\end{subfigure}
\caption{\label{Fig:memoryprogressionscopy}Memory progressions of element and vectorized copies}
\end{figure}

\newpage

As the optimized code produces a different result than the element-wise copy,
the compiler cannot just insert vector operations as it would make this specific invocation incorrect.
LLVM deals with this problem by adding \textit{runtime pointer checks}\footnote{\url{https://llvm.org/docs/Vectorizers.html\#runtime-checks-of-pointers}}.

An excerpt of the generated RV32IV assembly code which includes this check for our \texttt{copy} function is given in listing \ref{lst:runtimepointercheck}.
In lines 3 and 4 the end address of the \texttt{src} (\texttt{a3}) and \texttt{dst} (\texttt{a4}) arrays are computed.
Lines 5 and 6 determine whether there is overlap between them, by looking at the start and end addresses of both arrays.
If we assume the array from example \ref{Fig:memorycopy2} starts at addresses 0,
the overlap is detected because $2 < 3$ (the \texttt{dst} array begins before \texttt{src} ends)
and $0 < 10$ (the \texttt{src} array begins before \texttt{dst} ends).
Upon detecting overlap, the function immediately jumps to \texttt{.LBB0\_7}, after which it will perform the element-wise copy.
Otherwise, the function does not jump and continues, utilizing the vector operations
where possible.

If the programmer annotates the \texttt{src} and \texttt{dst} pointer declarations
with restrict, such runtime checks are omitted by the compiler.
In fact, the only difference in the generated code
when the restrict type qualifiers are present, is the omission of this specific code snippet (besides some renaming of labels).


\begin{code}
\begin{minted}[escapeinside=||,mathescape=true,linenos]{gas}
    ...
    slli    a3, a2, 2           # a3 := (n * 4) 
    add     a4, a1, a3          # a4 := dst + a3
    add     a3, a3, a0          # a3 := a3 + src
    sltu    a3, a1, a3          # a3 := (a1 < a3) ? 1:0
    sltu    a4, a0, a4          # a4 := (a0 < a4) ? 1:0
    and     a3, a3, a4          # a3 := a3 & a4
    li      a6, 0               # a6 := 0
    bnez    a3, .LBB0_7         # (a3 != 0) ? goto .LBB0_7
    ...
\end{minted}
\captionof{listing}{Runtime overlapping check in RV32IV assembly, emitted by Clang 14.0.0}
\label{lst:runtimepointercheck}
\end{code}

% RISC-V rv32gc clang 14.0.0
% -O3 -ffreestanding -fno-strict-aliasing -march=rv32gcv -target riscv32 -mllvm --riscv-v-vector-bits-min=256


\section{ISO/IEC definition}\label{section:iso-definition}
Section 6.7.3.1 of the ISO/IEC C11 standard \cite{ISO:2018:III} describes the ``formal definition" of restrict.
As we indicated in the introductory text, this definition can be considered quite complicated and unclear
with both implementors and users of the programming language having trouble interpreting its exact meaning.
Several proposals for a new or updated definition \cite{johnsonclarifying2018, provmacdonald2022, defectr2macdonald2024, semanticsgustedt2024}
even two decades after its original publication (the definition has remained unchanged since) further substantiate this claim.
% This is further supported by recent attempts from the community to rephrase (parts of) the formal definition.
% For example, MacDonald and Homer \cite{provmacdonald2022} propose a ``provenance-style specification"
% in the style of N3005, which is a technical specification attempting to (re)define more subtle aspects of the C memory model
% in which pointers to identical numerical addresses can be distinguished \cite{provmemgustedt2022}.
% Even more recently, Gustedt \cite{semanticsgustedt2024} describes how the mixup of semantic concepts makes the formal definition almost 
% impossible to comprehend and proposes a new definition.

In this thesis, we stated that fixing the ``formal definition" of restrict in the standard or analyzing the proposals for 
new definitions in natural language in detail are explicit non-goals.
However, we still want the operational semantics we develop in this thesis to relate closely to the \textit{intended}
semantics of restrict in the standard (especially for contexts in which its meaning is clear).
Therefore, we will explain our interpretation without reciting the entire definition.
This interpretation will be used in chapter \ref{chap:cink} to argue what the expected semantics of a program 
utilizing restrict with respect to the standard should be.

The standard definition starts by outlining the context of a restrict type qualifier by defining several variables.
$D$ is a declaration of an ordinary identifier.
$P$ is an object which can ``be designated" as a restrict qualified pointer to type $T$ through $D$.
It is not explicitly defined what it means to be able to designate an object as a restrict qualified pointer through a declaration.
We assume here that either some part of the type of $D$ has type \mintinline{c}{T* restrict}, or some part of a member type of 
a composite type has type \mintinline{c}{T* restrict}.
For example, the declaration $D1$ in \mintinline{c}{int* restrict *D1 = ...;} allows to designate such a restrict qualified object through \mintinline{c}{*D1 = ...;}
and the declaration $D2$ in \mintinline{c}{struct s {int* restrict p;} D2;} allows to designate such an object through \mintinline{c}{D2.p = ...;}.
Finally, $B$ denotes the block scope in which $D$ appears (which is \texttt{main} if $D$ has storage class extern or does not occur in a block).
We refer to this block as the \textit{restrict block}.

To demonstrate how these variables are constructed for a program, 
consider the function \texttt{foo} below, whose associated function block scope is named \scope{foo}.
The variables for this program are instantiated to
$D$ = \mintinline{c}{int* restrict p} (\ie $p$ is the ordinary identifier),
$P$ = $p$, $T = $ \mintinline{c}{int} and $B$ = \scope{foo}.

\begin{code}
\begin{minted}[escapeinside=||,mathescape=true,linenos]{c}
// Scope $\scope{foo}$
void foo(int* restrict p) {
...
}
\end{minted}
\end{code}

Furthermore, $L$ is any lvalue which has $\&L$ ``based on" $P$ and $X$ is the object designated by $L$.
The idea of the ``based on" definition was previously explained in the introduction.
Basically, a pointer expression $E$ is based on $P$ if it depends directly on the value of $p$ or on another pointer expression which is based on $P$.
We will use the phrase ``$L$ is \textit{derived from} from $P$" to refer to lvalues $L$ which have $\&L$ based on $P$. 

The term \textit{lvalue} is used to refer to an expression which may occur at the left-hand side of the
assignment operator and has an actual address in memory.
% All other expressions are \textit{rvalue expression}s, and all lvalue expressions are also rvalue expressions.


The official definition of ``based on'' in the standard is quite obscure.
MacDonald \etall proposed a new definition for the first time in 2022 \cite{defectmacdonald2022},
which was recently superseded by ``a more straightforward fix" \cite{defectr2macdonald2024}.

Having set the context, the standard defines two situations which can lead to \textit{undefined behavior}
(programs for which anything may happen):

\begin{enumerate}
    \item During execution of $B$, $L$ is used to access $X$ and $X$ is modified by any means.
    Any modification of $X$ is considered to also modify $P$.
    $T$ shall not be const-qualified.
    All other lvalues used to access $X$ must have their address based on $P$, or the program has undefined behavior.
    \item If $P$ is assigned the value of a pointer expression $E$ that is based on
    another restricted pointer object associated with block $B2$, then either the execution of $B2$ shall begin
    before the execution of $B$ or the execution of $B2$ shall end prior to the assignment.
\end{enumerate}

In this thesis, we focus on a semantics which is able to detect point $1$, \ie the dynamic semantics
of memory accesses using lvalues based on a restrict qualified pointer.
The quadruple $(D, P, X, B)$ refers to the previously defined variables and will
be instantiated for specific programs later on in this thesis.
Point $2$ is out of scope for this thesis and will be briefly addressed in chapter \ref{chap:conclusion} on future work.
