\chapter{Introduction}\label{chap:introduction}
Despite being over five decades old C remains a widely used general purpose programming language,
recently ranking fourth in the IEEE Spectrum 2023 index \cite{ieee2023} and second in the TIOBE January 2024 index \cite{tiobe2024}.
An important reason for this is \textit{performance}: the language provides fine-grained control over memory and other hardware, which makes it especially suitable
for the development of system software.

This control comes with responsibilities because a compiler does not ensure
that a program which successfully compiles also complies with all the obligations imposed by the ISO/IEC standard.
A program which violates these obligations is said to have \textit{undefined behavior} \cite[3.4.3]{ISO:2018:III},
``for which the International Standard imposes no requirements'', \ie a compiler may do \textit{anything} for such a program.
Undefined behavior (UB) effectively serves as a \textit{contract} between the programmer and the compiler.
The programmer is responsible for ensuring a program is free of undefined behavior, \eg by intuitive reasoning
or a static analysis/verification tool.
The compiler may in turn assume a program does not contain undefined behavior, which allows it to generate more efficient code
because it has to perform fewer checks.
For example, dereferencing out-of-bounds array pointers is considered undefined behavior, hence the
compiler does not insert bound checking instructions.

If undefined behavior leads to a run-time error, \eg a termination signal from the Linux Kernel such as SIGSEGV, this can be considered 
a good thing as it could be detected during testing.
A more tricky situation arises when a program is seemingly running correctly, but produces an ``incorrect'' result
due to a bug inducing undefined behavior.
Wang \etall \cite{wang2012undefined} find that such bugs are ``tricky to identify, hard to detect and understand, leading to programmers brushing them off incorrectly as GCC bugs''.

One specific feature of C with a subtle semantics which may induce undefined behavior is the \textit{restrict} type qualifier,
first introduced in the ISO/IEC C99 standard~\cite[6.7.3.1]{ISO:2018:III}.
This type qualifier can only be applied to pointer types.
The intended use is for a programmer to hint a compiler that only the restrict pointer will be used to access the
object it points to.
This allows a compiler to assume that under certain conditions specific pointers do not \textit{alias}, \ie they point to different objects.
This information is used by the compiler to justify program optimizations.
% By using restrict, the programmer \textit{promises} to the compiler that the object pointed to by the qualified pointer will not be accessed
% by pointers ``not based" on said pointer, if the object is modified within that scope.
% Because the memory accesses performed by a program determine whether the usage of restrict is correct or leads to undefined behavior
% its semantics are \textit{dynamic}, \ie the actual execution of the program determines whether it is well-defined or not
% and a compiler cannot statically check this.

Consider the function \textcode{foo} on page \pageref{lst:typical-optimization-permitted-by-restrict} for a small example of an optimization permitted by restrict.
This function takes two restrict qualified integer pointers $p$ and $q$ as arguments,
consecutively stores 10 via $p$ and 11 via $q$, and returns the integer value $p$ points to.
Because restrict allows the compiler to assume that $p$ and $q$ do not alias in the context
of \textcode{foo}, it knows that the store via $q$ could not have modified the object $p$ points to.
This means that at line \ref{optimization-candidate}, the object where $p$ points to must still contain the value $10$. 
When compiling with optimization flag -O3, both GCC and Clang optimize line \ref{optimization-candidate} to simply return 10 instead of loading from $p$ due to this information.
This optimization saves only a single memory load, but some more interesting examples will be presented in chapter \ref{chapter:examples}. 

\begin{code}
\begin{minted}[escapeinside=||,mathescape=true,linenos,texcomments]{c}
int foo(int* restrict p, int* restrict q)
{   
    *p = 10;
    *q = 11;
    return *p; // Desired optimization: replace with $\texttt{return 10;}$ $\label{optimization-candidate}$
}
\end{minted}
% \captionof{listing}{A typical optimization permitted by restrict}
\label{lst:typical-optimization-permitted-by-restrict}
\end{code}

The conditions under which a compiler may assume specific pointers do not alias and whether a program is well-defined regard the actual \textit{usage} of a restrict pointer.
For a well-defined program, qualifying a pointer with restrict has the following (simplified) meaning:
\textbf{if} a pointer expression ``based on'' the restrict qualified pointer is used to \textit{access} the object it points to
\textbf{and} that object is also \textit{modified} in some way, \textbf{then} all accesses of that object must happen
via pointer expressions based on the restrict qualified object.
A pointer expression is said to be \textit{based on} a restrict qualified pointer if it depends on its value. 
For example, in the example program above the pointer expression $p$ is based on $p$ and the pointer expression $q$ is based on $q$.
The use of restrict is a \textit{promise} by the programmer and not checked by the compiler.
If the requirements are violated, for example if $p$ and $q$ would alias in the example program above, the program has undefined behavior.


\paragraph{Problem statement} \leavevmode\\
The simplified definition we described above is already quite complex, but the complete specification of restrict in the standard is even more tricky to say the least.
The text is heavily technical and suffers both from the imprecision of natural language and lack of clarity on the semantics in
specific contexts, \eg the meaning of nested restrict pointers.

Recently, several proposals for a new definition in the standard have been submitted.
Gustedt \cite{semanticsgustedt2024} gives a list of problems with the specification and points out that due to
``a delicate mix up of semantic concepts the semantics of restrict is almost impossible to comprehend from the specification".
MacDonald \etall \cite{provmacdonald2022, defectr2macdonald2024} identify a problem with the definition of ``based on''
which prohibits certain programs from being optimized.

The subtleties of the semantics are further substantiated by the limited amount of optimizations performed by compilers.
In LLVM, which serves as the compiler backend for \eg Clang, restrict is only supported on function arguments and therefore misses opportunities for optimizations.
An ongoing effort for a more complete restrict implementation, including but not limited to restrict qualified struct members, was started in 2019 under LLVM-dev RFC 135672 \cite{rfcdobbelaere2019}.
GCC maintains a ``meta-bug" to track issues related to restrict~\cite{gccbugzillabiener2011}.
Most of the open issues are about missed optimizations, suggesting that GCC also ignores restrict in several places.
In GCC 7.3.0 a bug was found which led to an incorrect optimization due to restrict \cite{johnsonclarifying2018}
and was later resolved.

Overall, the problem is that the standard fails to adequately describe the semantics of restrict while correct usage is crucial both
for avoiding undefined behavior and promoting code optimizations.

\paragraph{Approach} \leavevmode\\
A well-established technique which rigorously describes what programs are well-defined and what program constructs induce undefined behavior
is a \textit{formal semantics}.
There exists a vast landscape of formal semantics for C.
Already in 1998, Norrish formalized a large fragment of C89 in the HOL4 proof assistant \cite{norrish1998c}.
Several years later Leroy~\etall created a large formalization in the Coq proof assistant for their verified CompCert compiler \cite{leroy2006formal}.
They prove for specific compiler optimizations that the program semantics are preserved between compiler optimization passes \cite{leroy2009formal, leroy2016compcert}.
In 2015, Krebbers presented the CH\textsubscript{2}O formalization for a large fragment of C11 including the \textit{strict aliasing rule}, a feature closely related to restrict \cite{krebbers2015c}.
More recently, Memarian presented Cerberus, an executable semantics for a large fragment of the ``de facto'' C11 standard \cite{memarian2023cerberus}.

Although much work has been done, restrict is omitted by most formalizations.
To the best of our knowledge, the only formal semantics project for C which does incorporate restrict is \cink{}, an executable semantics by Ellison and Ro{\c{s}}u \cite{ellison2012executable}.  
More specifically, the extension to their semantics by Hathhorn \etall \cite{hathhorn2015defining} aims to capture the semantics of undefined behavior and also addresses restrict.
However, the paper on this extension includes only a single paragraph on restrict which makes it somewhat unclear what the exact behavior in non-trivial programs is.
Although their test suite contains multiple tests for restrict, they do not include some common situations in which restrict may
induce undefined behavior.

The \cink{} implementation uses the $\mathbb{K}$-framework \cite{roșu2010overview}, a rewrite-based
semantics framework, and spans several thousands of lines of rewrite rules and definitions.
We redevelop the restrict fragment of this implementation in a functional style in order to better understand it.
The redevelopment consists of an \textit{operational semantics}, a description of the interpretation of a program as a sequence of computational steps,
and an implementation of the semantics in an interpreter.
We evaluate the semantics under a more extensive test set and find six programs for which we argue the semantics is incorrect.
We propose refinements to the semantic domains and rules to fix these problems and test them with our interpreter.
Then, we incorporate the new semantics in a small sequential C-like language and describe the language
semantics as a big-step operational semantics.
The language limits the variable types to \textcode{I32} (32-bit signed integers), \textcode{Ptr(\tau)} and \textcode{Array(\tau, n)}.
Although simple, we argue this provides enough variety to define a relevant semantics for restrict.
Firstly, because the inner type of a composite type is irrelevant for the restrict semantics, 
we have omitted other integers types, floating points and other base types.
Secondly, supporting arrays provides the necessity for reasoning about objects containing multiple values,
due to which we have omitted other composite types.

The operational semantics of restrict we present in this thesis uses an inherently different style
than the \textit{axiomatic style} of the standard section on restrict.
In such an axiomatic description one needs to verify whether certain conditions are met in order to determine if a program is well-defined or has undefined behavior.
An operational semantics is considered more intuitive and better suitable for implementation \cite{rocsu2012towards}.
The interpreter allows one to systematically test a program for undefined behavior, which is a great advantage over both the current standard and the aforementioned proposals for a new definition in natural language.
Another advantage of a formal semantics over a definition in natural language is that it provides a way to construct formal proofs which can justify optimizations permitted by restrict,
although we will not do this in this thesis.

\paragraph{Contributions}
\begin{itemize}
    \item We redevelop the restrict fragment of the \cink{} semantics in a functional style.
    We evaluate this fragment of their semantics for C and give arguments for its incorrectness for six
    programs, in relation to the ISO/IEC standard and/or existing compiler optimizations (chapter \ref{chap:cink}).
    \item We propose refinements to fix the identified problems with the restrict semantics and
    integrate the new \textit{Crestrict} semantics into a small C-like language whose meaning is described as a big-step operational semantics (chapter \ref{chapt:improved-semantics} and \ref{chapter:crestrict}).
    \item We implement the new semantics in the eponymous \textsc{Crestrict} interpreter\footnote{Available as Zenodo artifact \cite{klappe_2024_11031862}}, which is able to detect
    undefined behavior induced by programs violating the restrict semantics (chapter \ref{chapt:evaluation}).
\end{itemize}

\paragraph{Non-goals} \leavevmode \\
We will not fix the definition of the standard in natural language.

\paragraph{Notations and conventions} \leavevmode\\ 
The type $\Optiontype{\tau}$ denotes the optional type.
It is either \textit{some} value of type $\tau$, or \textit{none} denoted by $\epsilon$.
The \errbind \ operator has the same meaning as in Rust and denotes the monadic bind for the optional type:
if the result of the preceding expression had a value, retrieve that value from the \textit{some} constructor.
Otherwise, the function returns immediately, propagating the \nonesym \ value.

Lists are denoted as cons lists, in which 
[] denotes the empty list, and $x : xs$ the list $xs$ with the element $x$
in front of it.
We use lists to denote stacks as well.
For example, ($x : . .. : [y]$) denotes the stack with the item $x$ on top and $y$ at the bottom.

Maps from $K$ to $V$ are denoted as partial functions $K \rightharpoonup V$.
The notation $M\set{k \leftarrow \Valvar}$ denotes the map $M$ with the value $\Valvar \in V$ at key $k \in K$
(which overwrites the previous value at $k$ if there was one).
The notation $\set{k \mapsto \Valvar}$ denotes the singleton map with the value $\Valvar$ at key $k$.
Element retrieval is denoted by $M(k)$, which gives the value $\Valvar$ at key $k$ or $\epsilon$ if the map had no value at $k$. 
The notation $M_1 \setminus M_2$ denotes the set $M_1$ with all elements from $M_2$ removed (if they were in $M_1$).

Records use the same notation as in Stacked Borrows \cite{jung2019stacked}.
That is, a record is denoted as $\left\{\begin{array}{l} A : \Intdomain, B : \Booldomain \end{array} \right\}$ (\eg a record
with members $A$ of type $\Intdomain$ and $B$ of type $\Booldomain$).
Updates of a record $Y := [A, B]$ are denoted as $Y' := Y \pseudowith [A := A']$ (\ie $Y'$ is the record $Y$ with
member $A$ updated to $A'$).

Finally, for simple tuple types such as $\Simplelocvar \in \Simpleloc := \Block \times \Offset$ we will
assume functions which return a specific component exist with an eponymous name to the component type.
For example, \textcode{block}(\Simplelocvar) denotes the first component of the \Simpleloc \ \Simplelocvar.

% Specific tools, such as Google Sanitizers \cite{googlesanitizers}, provide a more practical way for (dynamic) error detection in C programs.
% They work by instrumenting a program at compile time, and use this additional data to verify the absence of
% specific kinds of undefined behavior at runtime.
% Among others, they check for memory leaks and overflow bugs, but no 
% checks for restrict induced undefined behavior are available\footnote{\url{https://github.com/google/sanitizers/issues/1228}}.

% // problem statement
% The semantics of restrict are subtle and currently do not lend themselves well for a ``formal definition'' in natural language.
% A formal semantics for restrict is desirable for both implementors and programmers (\eg as a practical tool in the style of Google Sanitizers),
% as well as more extensive formal semantics projects (\eg to justify optimizations).
% The only existing work incorporating restrict, \textsc{kcc}, do not present their semantics for restrict in appropriate detail
% and have some unsubstantiated discrepancies with the ISO/IEC standard definition. 


% Undecidable \cite{ramalingam1994undecidability}

% \section{History}

% https://www.lysator.liu.se/c/dmr-on-noalias.html

% No tools https://github.com/google/sanitizers/issues/1228

% \section{Examples}
% This chapter describes some real-world examples of optimizations performed by compilers
% based on programmer-specified aliasing information, using the \texttt{restrict} keyword in C. 
% Unless specifically stated otherwise, assembly code targets the RISC-V 32-bit architecture with
% the vector extension enabled (hereafter referred to as \texttt{RV32IV}).

% \subsection{Load optimization}\label{subsec:example-load-optimization}
