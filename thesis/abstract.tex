\begin{abstract}
Even after more than two decades since the first introduction of the \textit{restrict} type qualifier in the
C programming language its definition in the ISO/IEC standard raises questions and confusion.
For example, recently several proposals for a new definition have been submitted to the international standardization working group WG14, who maintain the standard.
Restrict effectively serves as a \textit{promise} from the programmer to the compiler that a restrict qualified
pointer will not \textit{alias} with specific other pointers under certain conditions.
If the promise is broken the program is said to have \textit{undefined behavior}, which may lead to tricky bugs which are difficult to detect.

In this work we take a different direction than (re)defining restrict in natural language.
We present \textit{Crestrict}, an \textit{operational semantics} which refines the restrict fragment of the \cink{} semantics for undefined behavior by Hathhorn \etall
Our work redevelops the original \cink{} semantics for restrict in a functional style.
We find six programs for which we argue the semantics gives either too much or too little undefined behavior with respect
to the ISO/IEC standard and/or existing compiler optimizations.
From these programs we propose several refinements and integrate the new Crestrict semantics in a small C-like language. 
We also implement the semantics in an interpreter to be able to test a given C program for undefined behavior induced by restrict.
We evaluate Crestrict under an extensive test suite and show that it gives more undefined behavior for test programs for which the \cink{} semantics
gives too little undefined behavior and vice versa.

With this work we give an alternative resource for restrict, which we argue to be more complete than existing work.
In particular, we provide a way for C programmers to systematically
test whether a given program utilizing restrict induces undefined behavior.
In the future the semantics could be used to prove restrict related compiler optimizations to be correct.
\end{abstract}
