\chapter{Evaluation}\label{chapt:evaluation}
To be able to test whether a given Crestrict program contains undefined behavior due to 
violations of the restrict semantics, we have developed an interpreter which can run
such programs and dynamically track whether the operations it performs lead to undefined behavior.
This implementation makes the semantics \textit{executable} and is described in section \ref{section:implementation}. 
We have created a test suite composed of 96 test programs.
For each program we expect it to have either defined or undefined behavior according to the ISO/IEC standard definition of restrict.
Furthermore, all tests we classify as \textit{defined} do not lead to incorrect program outputs
when compiled and optimized by GCC and Clang.
The categorization and other details of this test suite are described in section \ref{section:feature-categorization}.

\section{Implementation}\label{section:implementation}
The \textsc{Crestrict} interpreter is written in the Rust programming language.
It utilizes the \textsc{lang-c} parser\footnote{Available at \url{https://github.com/vickenty/lang-c}} crate,
which parses C source files into an abstract syntax tree (AST).
This AST is converted into a Crestrict AST.
If any unsupported language constructs were used, an error is returned immediately.
The AST is then type checked.
If type checking failed an error is returned immediately.
The Crestrict program is then interpreted according to the operational rules,
resulting in either a result code or an error message indicating what problem occurred, \ie 
the execution aborts immediately upon detecting undefined behavior
due to memory operations or a violation of the restrict semantics. 

There are only a few differences between the implementation and the operational semantics presented in chapter
\ref{chapter:crestrict} which are worth mentioning.
None of these differences are fundamental for the semantics, but rather an implementation choice that had to be made.

Firstly, instead of an \textdom{$\Optiontype{\tau}$} type denoting failure, the implementation uses 
Rust's \quad \texttt{Result}\textlangle$\tau$, \textdom{String}\textrangle \ type. 
The \texttt{Err} constructor is annotated with a string providing information as to where an error occurred.

Secondly, the implementation accepts a slightly larger language than the one presented in section \ref{section:syntax}:
initializers for global and local variables (including function calls) are also supported.
The initializer expressions are evaluated in declaration order and their values are directly stored into the memory.
Importantly, initializer values do \textbf{not} alter the restrict stack (as is also the case for the \cink{} semantics).
This choice is supported by the standard definition, \ie an initializer does not \textit{modify} an object as there was
nothing beforehand.
Finally, declarations inside the initial statement of a for loop are also supported.
The variable declaration is moved into the list of variable declarations of the function 
(which may only occur at the beginning), and the initializer is treated is described above.

\section{Feature categorization}\label{section:feature-categorization}
To the best of our knowledge, there is no publically available test suite dedicated to
the restrict type qualifier (although tests do exist as part of other test suites).
Therefore, we created our own suite to evaluate \textsc{Crestrict} and \textsc{kcc} against
(and when possible we have included tests from other test suites).
The test suite only contains programs whose syntax is supported by the Crestrict language
and was created simultaneously with the refined restrict semantics, due to which \textsc{Crestrict} passes all tests.
All the examples which were presented in section \ref{sec:cink-incorrect} are included in the test suite.
% In this section we present the feature categorization of the test suite.
We note that we have no tests for assignments between restrict pointers,
as this was out of scope for this thesis (section \ref{section:iso-definition}).

The features are categorized based on the \textit{context} in which a restrict qualified pointer occurs,
which affects how the presence of the type qualifier should be interpreted.
For completeness, the restrict-related tests from the \textsc{kcc} interpreter and the
examples from the ISO/IEC standard are also included.
An overview of the different features and the number of defined/undefined tests is given in table \ref{table:feature-categorization}.
We will also point out for which kind of tests per feature \textsc{kcc} gives another result than \textsc{Crestrict}. 
For clarity, note that \textsc{kcc} is a much more mature tool in the sense that it is able
to detect a lot of different kinds of undefined behavior and accepts a much larger language as input.
\textsc{Crestrict} on the other hand has the sole focus of detecting undefined behavior which arises due to uses of restrict.

\begin{table}[H]
\centering
\noindent\begin{tabularx}{\textwidth}{lccX}
\toprule
\textbf{Feature}              & \textbf{\#DB}   & \textbf{\#UB} & \textbf{Description}                            \\ \midrule
\textbf{Aggregate type}       &      2          &       3       & Restrict pointers within an aggregate datatype. \\
\textbf{Global}               &      5          &       4       & Restrict pointers/lvalues with global scope.            \\        
\textbf{Nested}               &      9          &      29       & Nested pointers with the restrict type qualifier on one of the inner types. \\ \midrule
\textbf{KCC}                  &      1          &       2       & Tests for restrict from the \textsc{kcc} test suite\footnotemark. \\
\textbf{Standard}             &      2          &       1       & Restrict examples from the ISO/IEC C11 standard \cite{ISO:2018:III}. \\
\textbf{Other}                &     18          &      19       & Other tests representing possible useful programs for restrict. \\ \midrule
\textbf{Total}                &     38          &      58       & Total: 96
\end{tabularx}
\caption{Feature categorization}
\label{table:feature-categorization}
\end{table}
\footnotetext{\url{https://github.com/kframework/c-semantics/tree/master/tests}}

In the context of Crestrict, only a single \textbf{Aggregrate type} exists: the array data type.
The tests for this feature have one or more restrict pointers as part of an array.
The undefined behavior tests fail for \textsc{kcc}, due to the problem described in section \ref{subsec:indistinguishable-restrict-pointers}.

\textbf{Global} is the feature describing global restrict pointers and global pointers based on some restrict pointer.
One of the defined behavior tests fails for \textsc{kcc}, due to the problem described in section \ref{subsec:out-of-scope-bases}.

\textbf{Nested} is the feature describing nested pointers with the restrict type qualifier on one of the inner pointer types.
The defined behavior tests all pass for \textsc{kcc}, but ${\sim}44$\% of the undefined behavior tests fail due to the problem described
in section \ref{subsec:cink-nested-restrict-pointers}.

\textbf{KCC} has actually eight tests related to restrict.
Four of them test assignments between restrict pointers and are therefore omitted.
One test uses unsupported language features (const) and has therefore also been omitted.
Finally, one test was moved from undefined behavior to defined behavior in our suite,
because we do not support type casts.

\textbf{Standard} contains the example programs from the ISO/IEC standard.
Only the two combinations of example two and example three are included, as
the other examples use unsupported language features (struct) or demonstrate illegal
assignments between restrict pointers.
Furthermore, one of the defined behavior tests fails for \textsc{kcc} due to the problem
described in section \ref{subsec:aliasing-reads}.

Finally, \textbf{Other} contains test programs of both realistic use cases of
restrict and problematic programs not captured by the features listed above.
Two of the defined behavior tests fail for \textsc{kcc} due to the aliasing loads problem discussed in
section \ref{subsec:aliasing-reads} and two of the undefined behavior tests fail for \textsc{kcc}
due to the problem with the deferred check (section \ref{subsec:cink-inlining-semantics}) 
and call to free (section \ref{subsec:call-to-free}).


% \begin{table}[H]
% \begin{tabular}{@{}ll@{}}
% \toprule
% \textbf{Definition entry 6.7.3.1} & \textbf{Feature}                                                \\ \midrule
% \textbf{(2)}                      & \textbf{(a)} Identify the restrict block $B$                    \\
% \textbf{(3)}                      & \textbf{(a)} Identify lvalues based on some restrict object $P$ \\
% \multirow[t]{4}{*}{\textbf{(4)}}     & \textb f{(a)} Track $X$ is modified                                \\
%                                     & \textbf{(b)} Track $P$ is modified by modification of $X$                                \\
%                                     & \textbf{(c)} Allow aliasing reads                               \\
%                                     & \textbf{(d)} Basing restrict pointers on each other            
% \end{tabular}
% \end{table}


% \begin{itemize}
%     \item \textbf{2a}: the bases derived from the restrict qualified pointer must be

%     \item \textbf{3a}: the semantics are more strict than the standard.
% \end{itemize}






% Dynamic features:

% \begin{itemize}
%     \item Track \textit{based on} provenance (3)
%     \item Propagate \textit{based on} provenance through scopes (3)
%     \item Allow aliasing restrict without modification (4)
%     \item Modification of X implies modification of P (4 - 2)
%     \item Feature (4 - 2) is transitive
% \end{itemize}

