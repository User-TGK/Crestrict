\chapter{Conclusion and future work}\label{chap:conclusion}
In this work we have presented Crestrict, an operational semantics for the C99 restrict type qualifier
based on the restrict fragment of the \cink{} semantics.
We have seen arguments showing that the existing semantics inadequately models the type qualifier for six specific programs,
and proposed fixes which refine the semantics.
We incorporated the new semantics in a small but representative C-like language.
The test suite we created demonstrates that for all test programs for which we argued the \cink{} semantics gives too little undefined behavior
Crestrict gives more undefined behavior and vice versa. 
The interpreter makes the semantics executable and allows one to systematically test whether a given program utilizing restrict has undefined behavior.
This achieves our goal of providing an alternative resource for restrict, which is not subject to the complex and error-prone definition in natural language of the ISO/IEC standard.

\section{Future work}
There are various ways in which one could extend our work and we will describe some of them in this section.

\paragraph{Assignments between restrict pointers}
The first obvious extension is complete support for restrict.
As we pointed out in section \ref{section:iso-definition}
we have omitted the restriction for assignments between restrict pointers in our semantics.
This restriction basically states that only ``outer-to-inner" assignments 
between restrict pointers declared in nested blocks are well-defined \cite[6.7.3.1, p4]{ISO:2018:III}.
For example, in the code below the assignments of $q$ to $p$
and $r$ to $p$ induce undefined behavior, but the assignment of $p$ to $r$ is well-defined.
In this thesis, we have manually ensured that this restriction is not violated by
any of the example programs, to make sure it does not conflict with the semantics we have presented for restrict.

\begin{code}
\begin{minted}[escapeinside=||,mathescape=true]{c}
int* restrict p;
int* restrict q;
p = q; // Undefined behavior
{
    int* restrict r = p; // Well-defined
    p = r;               // Undefined behavior
}
\end{minted}
\end{code}

\paragraph{Extend to a larger language}
The language considered in this thesis is relatively small and has no support for \eg structs and integer pointer casts.
We expect incorporating structs and supporting restrict pointers as struct members should be
compatible with our semantics due to the refinement for array types (section \ref{sec:promoting-block-of-a-base}).
The integer-pointer casts problem seems more tricky, and an exploration combining our semantics and Memarian \etall \cite{memarian2019exploring}
could be interesting.
Furthermore, general type casts (\eg casting the restrict qualifier away from a pointer type) and const (which in conjunction with
restrict could possibly permit more optimizations) could be interesting.

\paragraph{Verification of program optimizations}
Another interesting road which can be explored is formally showing that the
proposed Crestrict semantics permits some desired compiler optimizations.
This could be done in the style of CompCert, \ie a method similar to the semantic preservation property
previously discussed in section \ref{sec:related-work}.
An alternative is Simuliris \cite{gaher2022simuliris}, a framework for the verification of concurrent program optimizations.
Simuliris has a strong focus on concurrency and establishes ``fair termination preservation''
for many optimizations, meaning that under a fair scheduler a terminating program is not allowed to be
turned into a diverging one.
Similar to CompCert, a correct program optimization means that the result of the optimized
program must be a possible result of the original program.
The framework demonstrates its effectiveness by instantiating it for a new concurrent Stacked Borrows (the Rust aliasing discipline previously 
explained in section \ref{section:rust}), and verifying the correctness of the original optimizations.

\paragraph{Program logic} The development of a program logic for Crestrict could be used to
formally verify that a program does not contain undefined behavior induced by restrict.

Krebbers \cite{krebbers2015c} defines a separation logic for CH\textsubscript{2}O core C,
and proved it sound with respect to the operational semantics.
Program verification employing this logic is done by proving a Hoare triple $\set{P} \ s \ \set{Q}$,
in which $s$ is a statement, $P$ the precondition and $S$ the postcondition.
Proving such a triple also ensures the absence of undefined behavior.


Louwrink \cite{louwrink2021separation} defines a separation logic for Stacked Borrows. 
Similar to Krebbers, he also states the adequacy theorem, from which it follows that if a Hoare triple can be derived
for a program it does not have undefined behavior.

\paragraph{Axiomatic semantics} As we explained in chapter \ref{chap:introduction}, the
ISO/IEC definition of restrict uses a rather axiomatic description.
We have chosen to create an operational semantics in this thesis as this type of semantics
lends itself better for implementation.
As a result, our semantics does not relate as closely to the standard description as
one could accomplish with an \textit{axiomatic semantics}.
Defining such a semantics and possibly show an equivalence relation with our semantics could therefore
be valuable.

\paragraph{Noalias} As we pointed out in section \ref{section:rust}, both Clang and rustc
aim to emit the LLVM \texttt{noalias} attribute for C's \textcode{T* restrict} and Rust's \textcode{\&mut \ T} types
within parameter declarations.
The LLVM description of noalias\footnote{\url{https://llvm.org/docs/LangRef.html\#parameter-attributes}} states that 
its definition is intentionally similar to the definition of C's restrict, but also points out some differences
(\eg annotating return values with noalias has a special meaning).  
Creating a formal semantics for noalias, or extending existing formal semantics for LLVM such as
Vellvm \cite{zhao2012formalizing} or K-LLVM \cite{li2020k} would therefore be valuable.
Having such a semantics would also pave the way to formally show whether Clang preserves
the program semantics by replacing all occurrences of restrict in function parameter declarations with noalias.
